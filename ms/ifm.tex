\documentclass[12pt,aps,prb,preprint]{revtex4}
\usepackage{pgf}

\def\ket#1{|#1\rangle}
\def\X{{\bf X}}
\def\A{{\bf A}}

\begin{document}

\title{A Classical Analog of Quantum Interaction-Free Measurement}

\author{Prasenjit Saha}
\email{psaha@physik.uzh.ch}
\affiliation{Physik-Institut, University of Zurich, Switzerland}
\author{Nick Kaiser}
\email{kaiser@hawaii.edu}
\affiliation{Institute for Astronomy, University of Hawaii at Manoa,
Honolulu, HI 96822}

\date{\today}

\begin{abstract}
Interaction-free measurement is usually discussed in quantum language and
refers to phenomena such as the presence of a photon absorber being
inferred without it actually absorbing any photons.  Here we make the
simple observation that there is an
exactly analogous situation in classical mechanics: If one has
a pair of identical coupled oscillators A \& B, and A is set in motion,
then one will see the strength of the oscillation of A slowly decrease
with time, and over longer times the energy will periodically `slosh' back
and forth between the two systems.  But if instead oscillator B is damped -- so
it is able to {\em absorb\/} energy -- the rate of decrease of the
energy in oscillator A will be suppressed.   
Thus one can infer the ability of oscillator B to 
absorb energy by observing the behaviour of oscillator A without, in
the limit that the damping is strong, there actually being any
energy absorbed.
% We review the physics and its realization in interferometric
% experiments.  We then consider a simple classical system: two coupled
% oscillators, one of which is damped; the energy dissipation can be
% made arbitrarily small by making the damping large enough.  The
% phenomenon can be considered analogous to interaction-free
% measurement.
\end{abstract}

\maketitle

\section{Introduction}

Many quantum paradoxes have been explored theoretically and
experimentally in recent years.  Most involve entangled states: the
famous EPR paradox is an old example, and quantum computing is a more
recent development\cite{2007qucosc.book}.  Entanglement is considered
an intrinsically quantum phenonemon with no classical analog.

There are also counter-intuitive quantum phenonema not involving
entanglement.  Imagine a photon, initially in a ``vertical'' polarized
state.  It is sent through a Faraday rotator, and then through a
vertically polarizing sheet.  The former rotates the polarization
plane towards ``horizontal'' by a small angle, say $\pi/(2N)$, where $N$
is large.  The latter effectively measures the polarization state in
the vertical/horizontal basis; if the result comes out horizontal, the
photon is absorbed, otherwise it goes through in a vertical
eigenstate.  Since $N$ is large, the chances of being absorbed are
small.  More specifically, after the rotation the {\em amplitude\/}
to be in the horizontal state will be of order $1/N$ so the probability to be
in that mode -- and therefore the probability of absorption -- is of order $1/N^2$.
Thus, most likely, the photon will get slightly rotated and then
reset to vertical.  If the photon is re-routed back to the apparatus
$N$ times, it will still most likely emerge vertically polarized.  The
rotator and the polarizer appear to cancel each other out.  The
general phenomenon is known as the quantum Zeno
effect,\cite{PhysRevA.41.2295} but it can also be described in
everyday language, in terms of waking and sleeping
states.\cite{orzel2009teach}

What happens if we now remove the polarizer?  Clearly, the $N$ Faraday
rotations will then convert the photon to a horizontal state.  Thus, the net
effect of the polarizer is to rotate a photon from
horizontal to vertical.  Despite the fact that the polarizer is really just an
absorber of horizontal photons, it can, within this experimental set-up, have
the effect of rotating photons instead of absorbing them.  This is the essence
of interaction-free measurement; we can infer, or `measure' the presence or absence
of the polarizer by measuring the polarization.  If the polarization is rotated
we know that the absorber cannot have been present -- and there is then of course
no absorption -- while if the polarization is not changed then we know that
the absorber was present, but again, at least in the limit $N \rightarrow \infty$,
no absorption actually takes place.

Since the initial suggestion
(essentially the $N=1$ case)\cite{1993FoPh...23..987E} and
generalization to the now-standard form, together with the first
experiments\cite{PhysRevLett.74.4763} interaction-free measurement has
been the subject of many
demonstrations\cite{voorthuysen:1504,2006JPhB...39.3177N} and
reviews,\cite{deweerd:272} and applications aiming towards medical
imaging are starting to appear\cite{PhysRevA.80.040902}.  (An older
usage of similar
terminology\cite{springerlink:10.1007/BF01327019,dicke:925} refers to
a situation where a non-detection enables an inference.)

In this paper, we relate the above simple picture on the one hand to
the standard description of interaction-free measurement in terms of
an interferometric setup, and on the other hand to a classical system
of coupled oscillators.

\section{Quantum/Wave-Mechanical formulation}

The above thought experiment involving rotation the polarization plane
of photon is one simple way to approach the concept of
interaction-free measurement, but not the only one.  Another
well-known description, which more closely resembles some of the
actual experiments, is in terms of a Mach-Zehnder interferometer.  The
horizontal and vertical polarization states are replaced by the two
arms of the interferometer.  Figure~\ref{fig:mz} schematically shows
the $N=1$ case; the general case would be a concatenation of such
setups.

With these concrete descriptions in mind, let us formalize the process
of interaction-free measurement.  We write the two eigenstates in the
notation of qbits
\begin{equation}
\ket0 \equiv \left( \begin{array}{c} 1 \\ 0 \end{array} \right)  \qquad
\ket1 \equiv \left( \begin{array}{c} 0 \\ 1 \end{array} \right)
\end{equation}
and
\begin{equation}
\X \equiv \left( \begin{array}{cc} 0 & 1 \\ 1 & 0 \end{array} \right)
\end{equation}
denotes a swap operator.  A beam splitter is given by the unitary operator
\begin{equation}
\exp\left[i\left({\textstyle\frac\pi2-\epsilon}\right)\right] \X \equiv
\sin\epsilon + i\cos\epsilon\,\X
\end{equation}
Then $\cos^2\epsilon$ is the reflectivity.  A mirror is simply $i\X$.

The consecutive operations of beam-splitting, reflection and beam
splitting can be written as
\begin{equation}
     (\sin\epsilon + i\cos\epsilon\,\X)
     \; i\X \;
     (\sin\epsilon + i\cos\epsilon\,\X)
\end{equation}
which simplifies to
\begin{equation}
     -(\sin2\epsilon + i\cos2\epsilon\,\X) \,.
\end{equation}
A cascade with beam-splitter followed by mirror $N-1$ times is
\begin{equation}
     \left[ (\sin\epsilon + i\cos\epsilon\,\X) \; i\X \right]^{N-1} \;
     (\sin\epsilon + i\cos\epsilon\,\X)
\end{equation}
which simplifies to
\begin{equation}
     (-1)^{N-1} (\sin N\epsilon + i\cos N\epsilon\,\X)  \,.
\end{equation}
If $N\epsilon=\pi/2$ this is identity, apart from a phase factor.

Now put an absorber in one of the arms.  Then we will have
\begin{equation}
     \left[ (\sin\epsilon + i\cos\epsilon\,\X) \; i\A \right]^{N-1} \;
     (\sin\epsilon + i\cos\epsilon\,\X)
  \qquad
  \A \equiv \left( \begin{array}{cc} 0 & 1 \\ 0 & 0 \end{array} \right)
\end{equation}
Working out the algebra gives
\begin{equation}
(-1)^{N-1} \, \cos^N\!\epsilon \,
\left( \begin{array}{cc} \tan\epsilon & -i\tan^2\epsilon \\
                         i            & \tan\epsilon 
       \end{array} \right)
\end{equation}
If $\epsilon$ is small (very slightly transmissive) this operator
changes $\ket0$ to $\ket1$.

The discussion above uses the quantum mechanical language of
`states', but the behaviour of such an interferometer can, of course,
but understood entirely as a purely classical wave obeying
Maxwell's equations.

\section{The classical analogy}

To flesh out the observation of the abstract, 
consider two coupled oscillators, one of them damped.
\begin{equation}
\begin{array}{l}
\ddot x + x + \kappa y = 0 \\
\ddot y + y + \kappa x + \gamma y = 0
\end{array}
\end{equation}
In the absence of damping ($\gamma=0$), it is easy to verify that
\begin{equation}
\begin{array}{l}
2x = \cos(\sqrt{1+\kappa}\,t) + \cos(\sqrt{1-\kappa}\,t)  \\
2y = \cos(\sqrt{1+\kappa}\,t) - \cos(\sqrt{1-\kappa}\,t)
\end{array}
\end{equation}
is a solution for the initial conditions
\begin{equation}
x(0) = 1, y(0) = \dot x(0) = \dot y(0) = 0 \,.
\end{equation}
If the coupling is weak ($\kappa\ll1$) the solution takes the form
\begin{equation}
\begin{array}{l}
x = \cos(\frac12\kappa t) \cos t \\
y = -\sin(\frac12\kappa t) \sin t \\
\end{array}
\end{equation}
and the amplitude sloshes from $x$ to $y$.  The weaker the coupling,
the slower the transfer of energy between the oscillators.

With damping, easiest to solve using Laplace transform. For the
initial conditions $x(0)=1$ and $\dot x(0) = y(0) = \dot y(0) = 0$ we
have
\begin{equation}
\begin{array}{l}
s^2 X - s + X + \kappa Y = 0 \\
s^2 Y + Y + \kappa X + \gamma s Y = 0
\end{array}
\end{equation}
For strong damping, we have to $O(1/\gamma)$
\begin{equation}
\begin{array}{l}
\displaystyle
X = \frac s{s^2+1} + \frac{\kappa^2}\gamma \frac1{(s^2+1)^2} \\
\displaystyle
Y = -\frac\kappa\gamma \frac1{s^2+1}
\end{array}
\end{equation}
Solution to $O(1/\gamma)$
\begin{equation}
\begin{array}{l}
\displaystyle
x = \cos t + \frac{\kappa^2}{2\gamma} (\sin t - t\cos t) \\
\displaystyle
y = -\frac\kappa\gamma \sin t
\end{array}
\end{equation}
Amplitude damping rate $\propto\kappa^2/\gamma$.

For mechanical oscillators, it may be counter-intuitive that more
damping means less energy dissipation.  In the electrical case (an LC
circuit coupled to an RLC circuit) it is obvious that increasing
resistance reduces dissipation.

\bibliography{ifm.bib}

\newpage

\def\vmargin{\hrule height 1cm width 0pt}


\def\pgfbeamsplitter(#1,#2)#3{\pgfputat{\pgfxy(#1,#2)}
                              {\pgfbox[center,center]#3
                               \pgfxyline(-1.1,0)(0,1.1)
                               \pgfxyline(0,1.1)(1.1,0)
                               \pgfxyline(1.1,0)(0,-1.1)
                               \pgfxyline(0,-1.1)(-1.1,0)}}

\def\pgfbeamsplitter(#1,#2)#3{\pgfputat{\pgfxy(#1,#2)}
                              {\pgfbox[center,center]#3
                               \pgfxyline(-1.1,0)(0,1.1)
                               \pgfxyline(0,1.1)(1.1,0)
                               \pgfxyline(1.1,0)(0,-1.1)
                               \pgfxyline(0,-1.1)(-1.1,0)}}


\def\semireflecting{$\hbox{beam}\atop\hbox{splitter}$}
\def\arbireflecting{$e^{i\alpha\X}$}
\def\nearlytransparent{$e^{i\left(\frac\pi2-\epsilon\right)\X}$}

\def\pgfsimpmirror(#1,#2){\pgfputat{\pgfxy(#1,#2)}
                      {\pgfbox[center,center]{M}
                      \pgfxyline(-.4,-.4)(.4,-.4)
                      \pgfxyline(.4,-.4)(.4,.4)
                      \pgfxyline(.4,.4)(-.4,.4)
                      \pgfxyline(-.4,.4)(-.4,-.4)}}

\def\pgfmirror(#1,#2){\pgfputat{\pgfxy(#1,#2)}
                      {\pgfbox[center,center]{$i\X$}
                      \pgfxyline(-.4,-.4)(.4,-.4)
                      \pgfxyline(.4,-.4)(.4,.4)
                      \pgfxyline(.4,.4)(-.4,.4)
                      \pgfxyline(-.4,.4)(-.4,-.4)}}
\def\pgfabsorber(#1,#2){\pgfputat{\pgfxy(#1,#2)}
                      {\pgfbox[center,center]{$\bf 0$}
                      \pgfxyline(-.4,-.4)(.4,-.4)
                      \pgfxyline(.4,-.4)(.4,.4)
                      \pgfxyline(.4,.4)(-.4,.4)
                      \pgfxyline(-.4,.4)(-.4,-.4)}}

\begin{figure}
\begin{center}
\vmargin
\begin{pgfpicture}
\pgfbeamsplitter(0,0)\semireflecting
\pgfsimpmirror(2,2) 
\pgfsimpmirror(2,-2)
\pgfbeamsplitter(4,0)\semireflecting
\pgfsetendarrow{\pgfarrowsingle}
\pgfxyline(-1.5,-1.5)(-.7,-.7) 
\pgfxyline(.7,.7)(1.5,1.5)
\pgfxyline(.7,-.7)(1.5,-1.5)
\pgfxyline(2.5,1.5)(3.3,.7) 
\pgfxyline(2.5,-1.5)(3.3,-.7)
\pgfxyline(4.7,.7)(5.5,1.5) 
\end{pgfpicture}
\vmargin\vmargin\vmargin
\begin{pgfpicture}
\pgfbeamsplitter(0,0)\semireflecting
\pgfsimpmirror(2,2) 
\pgfsimpmirror(2,-2)
\pgfbeamsplitter(4,0)\semireflecting
\pgfsetendarrow{\pgfarrowsingle}
\pgfxyline(-1.5, 1.5)(-.7, .7) 
\pgfxyline(.7,.7)(1.5,1.5)
\pgfxyline(.7,-.7)(1.5,-1.5)
\pgfxyline(2.5,1.5)(3.3,.7) 
\pgfxyline(2.5,-1.5)(3.3,-.7)
\pgfxyline(4.7,-.7)(5.5,-1.5) 
\end{pgfpicture}
\vmargin\vmargin\vmargin
\begin{pgfpicture}
\pgfbeamsplitter(0,0)\semireflecting
\pgfabsorber(2,2) 
\pgfsimpmirror(2,-2)
\pgfbeamsplitter(4,0)\semireflecting
\pgfsetendarrow{\pgfarrowsingle}
\pgfxyline(-1.5,-1.5)(-.7,-.7) 
\pgfxyline(.7,.7)(1.5,1.5)
\pgfxyline(.7,-.7)(1.5,-1.5)
\pgfxyline(2.5,-1.5)(3.3,-.7)
\pgfxyline(4.7,.7)(5.5,1.5) 
\pgfxyline(4.7,-.7)(5.5,-1.5) 
\end{pgfpicture}
\end{center}
\caption{\label{fig:mz} Schematic representation of interaction-free
  measurement. Each diagram stands for a Mach-Zehnder set up so that
  both arms are exactly equal.  In the upper and middle diagrams, a
  photon sent axially into one arm emerges parallel to the original
  direction.  In the lower diagram, one of the mirrors is replaced by
  an absorber, and as a result, photons can emerge perpendicular to
  their original direction.  Such photons effectively detect the
  absorber without getting absorbed.}
\end{figure}

\begin{figure}
\begin{center}
\vmargin
\begin{pgfpicture}
\pgfbeamsplitter(0,0)\nearlytransparent 
\pgfmirror(2,2) 
\pgfmirror(2,-2)
\pgfbeamsplitter(4,0)\nearlytransparent 
\pgfsetendarrow{\pgfarrowsingle}
\pgfxyline(-1.5,-1.5)(-.7,-.7) 
\pgfxyline(.7,.7)(1.5,1.5)
\pgfxyline(.7,-.7)(1.5,-1.5)
\pgfxyline(2.5,1.5)(3.3,.7) 
\pgfxyline(2.5,-1.5)(3.3,-.7)
\pgfxyline(4.7,.7)(5.5,1.5)
\pgfxyline(4.7,-.7)(5.5,-1.5)
\pgfputat{\pgfxy(-1.7,-1.7)}{\pgfbox[right,center]{$\ket0$}} 
\pgfputat{\pgfxy(5.7, 1.7)}{\pgfbox[left,center]
         {$-\sin2\epsilon\;\ket0$}}
         \pgfputat{\pgfxy(5.7,-1.7)}{\pgfbox[left,center]
         {$-i\cos2\epsilon\;\ket1$}}
\end{pgfpicture}
\vmargin\vmargin\vmargin
\begin{pgfpicture}
\pgfbeamsplitter(0,0)\nearlytransparent 
\pgfmirror(2,2) 
\pgfmirror(2,-2)
\pgfbeamsplitter(4,0)\nearlytransparent 
\pgfsetendarrow{\pgfarrowsingle}
\pgfxyline(-1.5, 1.5)(-.7, .7) 
\pgfxyline(.7,.7)(1.5,1.5)
\pgfxyline(.7,-.7)(1.5,-1.5)
\pgfxyline(2.5,1.5)(3.3,.7) 
\pgfxyline(2.5,-1.5)(3.3,-.7)
\pgfxyline(4.7,.7)(5.5,1.5)
\pgfxyline(4.7,-.7)(5.5,-1.5)
\pgfputat{\pgfxy(-1.7, 1.7)}{\pgfbox[right,center]{$\ket1$}} 
\pgfputat{\pgfxy(5.7, 1.7)}{\pgfbox[left,center]
         {$-i\cos2\epsilon\;\ket0$}}
          \pgfputat{\pgfxy(5.7,-1.7)}{\pgfbox[left,center]
         {$-\sin2\epsilon\;\ket1$}}
\end{pgfpicture}
\vmargin\vmargin\vmargin
\begin{pgfpicture}
\pgfbeamsplitter(0,0)\nearlytransparent 
\pgfabsorber(2,2) 
\pgfmirror(2,-2)
\pgfbeamsplitter(4,0)\nearlytransparent 
\pgfsetendarrow{\pgfarrowsingle}
\pgfxyline(-1.5,-1.5)(-.7,-.7) 
\pgfxyline(.7,.7)(1.5,1.5)
\pgfxyline(.7,-.7)(1.5,-1.5)
\pgfxyline(2.5,-1.5)(3.3,-.7)
\pgfxyline(4.7,.7)(5.5,1.5)
\pgfxyline(4.7,-.7)(5.5,-1.5)
\pgfputat{\pgfxy(-1.7,-1.7)}{\pgfbox[right,center]{$\ket0$}} 
\pgfputat{\pgfxy(5.7, 1.7)}{\pgfbox[left,center]
         {$-\sin\epsilon\cos\epsilon\;\ket0$}}
          \pgfputat{\pgfxy(5.7,-1.7)}{\pgfbox[left,center]
         {$-i\cos^2\epsilon\;\ket1$}}
\end{pgfpicture}
\end{center}
\caption{\label{fig:qubits} Like Figure \ref{fig:mz} but annotated
  with states and operators.  A photon is denoted by $\ket0$ if moving
  towards upper right, and by $\ket1$ if moving towards lower right.
  Beam splitters and mirrors are unitary operators.  An absorber is a
  non-unitary operator, which zeros the $\ket0$ subspace.}
\end{figure}



\end{document}

