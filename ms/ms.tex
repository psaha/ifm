\documentclass[12pt,aps,prb,preprint]{revtex4}
\usepackage{pgf}

\def\ket#1{|#1\rangle}
\def\X{{\bf X}}
\def\A{{\bf A}}

\begin{document}

\title{Quantum interaction-free measurement and a classical analog}

\author{Prasenjit Saha}
\email{psaha@physik.uzh.ch}
\affiliation{Institute for Theoretical Physics, University of Zurich,
  Switzerland}
\author{Nick Kaiser}
\email{kaiser@hawaii.edu}
\affiliation{Institute for Astronomy, University of Hawaii at Manoa,
Honolulu, HI 96822}

\date{\today}

\begin{abstract}
Interaction-free measurement refers to quantum phenomena such as the
presence of a photon absorber being inferred without it absorbing any
photons.  We review the physics and its realization in interferometric
experiments.  We then consider a simple classical system: two coupled
oscillators, one of which is damped; the energy dissipation can be
made arbitrarily small by making the damping large enough.  The
phenomenon can be considered analogous to interaction-free
measurement.
\end{abstract}

\maketitle

\section{Introduction}

Many quantum paradoxes have been explored theoretically and
experimentally in recent years.  Most involve entangled states: the
famous EPR paradox is an old example, and quantum computing is a more
recent development\cite{2007qucosc.book}.  Entanglement is considered
an intrinsically quantum phenonemn classical analog.

There are also counter-intuitive quantum phenonema not involving
entanglement.  Imagine a photon, initially in a ``vertical'' polarized
state.  It is sent through a Faraday rotator, and then through a
vertically polarizing sheet.  The former rotates the polarization
plane towards ``horizontal'' by small angle, say $\pi/(2N)$, where $N$
is large.  The latter effectively measures the polarization state in
the vertical/horizontal basis; if the result comes out horizontal, the
photon is absorbed, otherwise it goes through in a vertical
eigenstate.  Since $N$ is large, the chances of being absorbed are
small.  Most likely, the photon will get slightly rotated and then
reset to vertical.  If the photon is re-routed back to the apparatus
$N$ times, it will still most likely emerge vertically polarized.  The
rotator and the polarizer appear to cancel each other out.  The
general phenomenon is known as the quantum Zeno
effect,\cite{PhysRevA.41.2295} but it can also be described in
everyday language, in terms of waking and sleeping
states.\cite{orzel2009teach}

What happens if we now remove the polarizer?  Clearly, the $N$ Faraday
rotations will take the photon to a horizontal state.  Thus, the net
effect of the polarizer is (most likely) to rotate a photon from
horizontal to vertical.  Recall that the polarizer is really just an
absorber of horizontal photons.  Yet it can, within a clever setup, be
made to rotate photons instead of absorbing them.  This is the essence
of interaction-free measurement.  Since the initial suggestion
(essentially the $N=1$ case)\cite{1993FoPh...23..987E} and
generalization to the now-standard form, together with the first
experiments\cite{PhysRevLett.74.4763} interaction-free measurement has
been the subject of many
demonstrations\cite{voorthuysen:1504,2006JPhB...39.3177N} and
reviews,\cite{deweerd:272} and applications aiming towards medical
imaging are starting to appear\cite{PhysRevA.80.040902}.  (An older
usage of similar
terminalogy\cite{springerlink:10.1007/BF01327019,dicke:925} refers to
a situation where a non-detection enables an inference.)

In this paper, we relate the above simple picture on the one hand to
the standard description of interaction-free measurement in terms of
an interferometric setup, and on the other hand to a classical system
of coupled oscillators.

\section{Quantum formulation}

The above thought experiment involving rotation the polarization plane
of photon is one simple way to approach the concept of
interaction-free measurement, but not the only one.  Another
well-known description, which more closely resembles some of the
actual experiments, is in terms of a Mach-Zehnder interferometer.  The
horizontal and vertical polarization states are replaced by the two
arms of the interferometer.  Figure~\ref{fig:mz} schematically shows
the $N=1$ case; the general case would be a concatenation of such
setups.

With these concrete descriptions in mind, let us formalize the process
of interaction-free measurement.  We write the two eigenstates in the
notation of cubits
\begin{equation}
\ket0 \equiv \left( \begin{array}{c} 1 \\ 0 \end{array} \right)  \qquad
\ket1 \equiv \left( \begin{array}{c} 0 \\ 1 \end{array} \right)
\end{equation}
and
\begin{equation}
\X \equiv \left( \begin{array}{cc} 0 & 1 \\ 1 & 0 \end{array} \right)
\end{equation}
denotes a swap operator.  A beam splitter is given by the unitary operator
\begin{equation}
\exp\left[i\left({\textstyle\frac\pi2-\epsilon}\right)\right] \X \equiv
\sin\epsilon + i\cos\epsilon\,\X
\end{equation}
Then $\cos^2\epsilon$ is the reflectivity.  A mirror is simply $i\X$.

The consecutive operations of beam-splitting, reflection and beam
splitting can be written as
\begin{equation}
     (\sin\epsilon + i\cos\epsilon\,\X)
     \; i\X \;
     (\sin\epsilon + i\cos\epsilon\,\X)
\end{equation}
which simplifies to
\begin{equation}
     -(\sin2\epsilon + i\cos2\epsilon\,\X) \,.
\end{equation}
A cascade with beam-splitter followed by mirror $N$ times is
\begin{equation}
     \left[ (\sin\epsilon + i\cos\epsilon\,\X) \; i\X \right]^N \;
     (\sin\epsilon + i\cos\epsilon\,\X)
\end{equation}
which simplifies to
\begin{equation}
     (-1)^{N-1} (\sin N\epsilon + i\cos N\epsilon\,\X)  \,.
\end{equation}
If $N\epsilon=\pi/2$ this is identity, apart from a phase factor.

Now put an absorber in one of the arms.  Then we will have
\begin{equation}
     \left[ (\sin\epsilon + i\cos\epsilon\,\X) \; i\A \right]^N \;
     (\sin\epsilon + i\cos\epsilon\,\X)
  \qquad
  \A \equiv \left( \begin{array}{cc} 0 & 1 \\ 0 & 0 \end{array} \right)
\end{equation}
Working out the algebra gives
\begin{equation}
(-1)^{N-1} \, \cos^N\!\epsilon \,
\left( \begin{array}{cc} \tan\epsilon & -i\tan^2\epsilon \\
                         i            & \tan\epsilon 
       \end{array} \right)
\end{equation}
If $\epsilon$ is small (very slightly transmissive) this operator
changes $\ket0$ to $\ket1$.

\section{A classical analogy}

Consider two coupled oscillators, one of them damped.
\begin{equation}
\begin{array}{l}
\ddot x + x + \kappa y = 0 \\
\ddot y + y + \kappa x + \gamma y = 0
\end{array}
\end{equation}
If $\gamma=0$ has solutions of the form
\begin{equation}
\begin{array}{l}
x = \cos(\sqrt{1+\kappa}\,t) + \cos(\sqrt{1-\kappa}\,t)  \\
y = \cos(\sqrt{1+\kappa}\,t) - \cos(\sqrt{1-\kappa}\,t)
\end{array}
\end{equation}
and the amplitude sloshes from one to the other.

With damping, easiest to solve using Laplace transform. For the
initial conditions $x(0)=1$ and $\dot x(0) = y(0) = \dot y(0) = 0$ we
have
\begin{equation}
\begin{array}{l}
s^2 X - s + X + \kappa Y = 0 \\
s^2 Y + Y + \kappa X + \gamma s Y = 0
\end{array}
\end{equation}
For strong damping, we have to $O(1/\gamma)$
\begin{equation}
\begin{array}{l}
\displaystyle
X = \frac s{s^2+1} + \frac{\kappa^2}\gamma \frac1{(s^2+1)^2} \\
\displaystyle
Y = -\frac\kappa\gamma \frac1{s^2+1}
\end{array}
\end{equation}
Solution to $O(1/\gamma)$
\begin{equation}
\begin{array}{l}
\displaystyle
x = \cos t + \frac{\kappa^2}{2\gamma} (\sin t - t\cos t) \\
\displaystyle
y = -\frac\kappa\gamma \sin t
\end{array}
\end{equation}
Amplitude damping rate $\propto\kappa^2/\gamma$.

For mechanical oscillators, it may be counter-intuitive that more
damping means less energy dissipation.  In the electrical case (an LC
circuit coupled to an RLC circuit) it is obvious that increasing
resistance reduces dissipation.

\bibliography{heap.bib}

\newpage

\def\vmargin{\hrule height 1cm width 0pt}


\def\pgfbeamsplitter(#1,#2)#3{\pgfputat{\pgfxy(#1,#2)}
                              {\pgfbox[center,center]#3
                               \pgfxyline(-1.1,0)(0,1.1)
                               \pgfxyline(0,1.1)(1.1,0)
                               \pgfxyline(1.1,0)(0,-1.1)
                               \pgfxyline(0,-1.1)(-1.1,0)}}

\def\pgfbeamsplitter(#1,#2)#3{\pgfputat{\pgfxy(#1,#2)}
                              {\pgfbox[center,center]#3
                               \pgfxyline(-1.1,0)(0,1.1)
                               \pgfxyline(0,1.1)(1.1,0)
                               \pgfxyline(1.1,0)(0,-1.1)
                               \pgfxyline(0,-1.1)(-1.1,0)}}


\def\semireflecting{$\hbox{beam}\atop\hbox{splitter}$}
\def\arbireflecting{$e^{i\alpha\X}$}
\def\nearlytransparent{$e^{i\left(\frac\pi2-\epsilon\right)\X}$}

\def\pgfsimpmirror(#1,#2){\pgfputat{\pgfxy(#1,#2)}
                      {\pgfbox[center,center]{M}
                      \pgfxyline(-.4,-.4)(.4,-.4)
                      \pgfxyline(.4,-.4)(.4,.4)
                      \pgfxyline(.4,.4)(-.4,.4)
                      \pgfxyline(-.4,.4)(-.4,-.4)}}

\def\pgfmirror(#1,#2){\pgfputat{\pgfxy(#1,#2)}
                      {\pgfbox[center,center]{$i\X$}
                      \pgfxyline(-.4,-.4)(.4,-.4)
                      \pgfxyline(.4,-.4)(.4,.4)
                      \pgfxyline(.4,.4)(-.4,.4)
                      \pgfxyline(-.4,.4)(-.4,-.4)}}
\def\pgfabsorber(#1,#2){\pgfputat{\pgfxy(#1,#2)}
                      {\pgfbox[center,center]{$\bf 0$}
                      \pgfxyline(-.4,-.4)(.4,-.4)
                      \pgfxyline(.4,-.4)(.4,.4)
                      \pgfxyline(.4,.4)(-.4,.4)
                      \pgfxyline(-.4,.4)(-.4,-.4)}}

\begin{figure}
\begin{center}
\vmargin
\begin{pgfpicture}
\pgfbeamsplitter(0,0)\semireflecting
\pgfsimpmirror(2,2) 
\pgfsimpmirror(2,-2)
\pgfbeamsplitter(4,0)\semireflecting
\pgfsetendarrow{\pgfarrowsingle}
\pgfxyline(-1.5,-1.5)(-.7,-.7) 
\pgfxyline(.7,.7)(1.5,1.5)
\pgfxyline(.7,-.7)(1.5,-1.5)
\pgfxyline(2.5,1.5)(3.3,.7) 
\pgfxyline(2.5,-1.5)(3.3,-.7)
\pgfxyline(4.7,.7)(5.5,1.5) 
\end{pgfpicture}
\vmargin\vmargin\vmargin
\begin{pgfpicture}
\pgfbeamsplitter(0,0)\semireflecting
\pgfsimpmirror(2,2) 
\pgfsimpmirror(2,-2)
\pgfbeamsplitter(4,0)\semireflecting
\pgfsetendarrow{\pgfarrowsingle}
\pgfxyline(-1.5, 1.5)(-.7, .7) 
\pgfxyline(.7,.7)(1.5,1.5)
\pgfxyline(.7,-.7)(1.5,-1.5)
\pgfxyline(2.5,1.5)(3.3,.7) 
\pgfxyline(2.5,-1.5)(3.3,-.7)
\pgfxyline(4.7,-.7)(5.5,-1.5) 
\end{pgfpicture}
\vmargin\vmargin\vmargin
\begin{pgfpicture}
\pgfbeamsplitter(0,0)\semireflecting
\pgfabsorber(2,2) 
\pgfsimpmirror(2,-2)
\pgfbeamsplitter(4,0)\semireflecting
\pgfsetendarrow{\pgfarrowsingle}
\pgfxyline(-1.5,-1.5)(-.7,-.7) 
\pgfxyline(.7,.7)(1.5,1.5)
\pgfxyline(.7,-.7)(1.5,-1.5)
\pgfxyline(2.5,-1.5)(3.3,-.7)
\pgfxyline(4.7,.7)(5.5,1.5) 
\pgfxyline(4.7,-.7)(5.5,-1.5) 
\end{pgfpicture}
\end{center}
\caption{\label{fig:mz} Schematic representation of interaction-free
  measurement. Each diagram stands for a Mach-Zehnder set up so that
  both arms are exactly equal.  Accordingly, a photon sent axially
  into one arm emerges parallel to the original direction.  This is
  seen in the upper and middle diagrams.  In the lower diagram, one of
  the mirrors is replaced by an absorber, and as a result, photons can
  emerge parallel to their original direction.  Such photons
  effectively detect the absorber without getting absorbed.}
\end{figure}

\begin{figure}
\begin{center}
\vmargin
\begin{pgfpicture}
\pgfbeamsplitter(0,0)\nearlytransparent 
\pgfmirror(2,2) 
\pgfmirror(2,-2)
\pgfbeamsplitter(4,0)\nearlytransparent 
\pgfsetendarrow{\pgfarrowsingle}
\pgfxyline(-1.5,-1.5)(-.7,-.7) 
\pgfxyline(.7,.7)(1.5,1.5)
\pgfxyline(.7,-.7)(1.5,-1.5)
\pgfxyline(2.5,1.5)(3.3,.7) 
\pgfxyline(2.5,-1.5)(3.3,-.7)
\pgfxyline(4.7,.7)(5.5,1.5)
\pgfxyline(4.7,-.7)(5.5,-1.5)
\pgfputat{\pgfxy(-1.7,-1.7)}{\pgfbox[right,center]{$\ket0$}} 
\pgfputat{\pgfxy(5.7, 1.7)}{\pgfbox[left,center]
         {$-\sin2\epsilon\;\ket0$}}
         \pgfputat{\pgfxy(5.7,-1.7)}{\pgfbox[left,center]
         {$-i\cos2\epsilon\;\ket1$}}
\end{pgfpicture}
\vmargin\vmargin\vmargin
\begin{pgfpicture}
\pgfbeamsplitter(0,0)\nearlytransparent 
\pgfmirror(2,2) 
\pgfmirror(2,-2)
\pgfbeamsplitter(4,0)\nearlytransparent 
\pgfsetendarrow{\pgfarrowsingle}
\pgfxyline(-1.5, 1.5)(-.7, .7) 
\pgfxyline(.7,.7)(1.5,1.5)
\pgfxyline(.7,-.7)(1.5,-1.5)
\pgfxyline(2.5,1.5)(3.3,.7) 
\pgfxyline(2.5,-1.5)(3.3,-.7)
\pgfxyline(4.7,.7)(5.5,1.5)
\pgfxyline(4.7,-.7)(5.5,-1.5)
\pgfputat{\pgfxy(-1.7, 1.7)}{\pgfbox[right,center]{$\ket1$}} 
\pgfputat{\pgfxy(5.7, 1.7)}{\pgfbox[left,center]
         {$-i\cos2\epsilon\;\ket0$}}
          \pgfputat{\pgfxy(5.7,-1.7)}{\pgfbox[left,center]
         {$-\sin2\epsilon\;\ket1$}}
\end{pgfpicture}
\vmargin\vmargin\vmargin
\begin{pgfpicture}
\pgfbeamsplitter(0,0)\nearlytransparent 
\pgfabsorber(2,2) 
\pgfmirror(2,-2)
\pgfbeamsplitter(4,0)\nearlytransparent 
\pgfsetendarrow{\pgfarrowsingle}
\pgfxyline(-1.5,-1.5)(-.7,-.7) 
\pgfxyline(.7,.7)(1.5,1.5)
\pgfxyline(.7,-.7)(1.5,-1.5)
\pgfxyline(2.5,-1.5)(3.3,-.7)
\pgfxyline(4.7,.7)(5.5,1.5)
\pgfxyline(4.7,-.7)(5.5,-1.5)
\pgfputat{\pgfxy(-1.7,-1.7)}{\pgfbox[right,center]{$\ket0$}} 
\pgfputat{\pgfxy(5.7, 1.7)}{\pgfbox[left,center]
         {$-\sin\epsilon\cos\epsilon\;\ket0$}}
          \pgfputat{\pgfxy(5.7,-1.7)}{\pgfbox[left,center]
         {$-i\cos^2\epsilon\;\ket1$}}
\end{pgfpicture}
\end{center}
\caption{\label{fig:qubits} Like Figure \ref{fig:mz} but annotated
  with quantum states and operators.  A photon is denoted by $\ket0$
  if moving towards upper right, and by $\ket1$ if moving towards
  lower right.  Beam splitters and mirrors are unitary operators.  An
  absorber is a non-unitary operator, which zeros the $\ket0$
  subspace.}
\end{figure}



\end{document}

