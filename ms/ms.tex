\documentclass[12pt,aps,prb,preprint]{revtex4}
\usepackage{pgf}

\def\ket#1{|#1\rangle}
\def\X{{\bf X}}

\begin{document}

\title{Quantum interaction-free measurement and a classical analog}

\author{Prasenjit Saha}
\email{psaha@physik.uzh.ch}
\affiliation{Institute for Theoretical Physics, University of Zurich,
  Switzerland}
\author{Nick Kaiser}
\email{kaiser@hawaii.edu}
\affiliation{Institute for Astronomy, University of Hawaii at Manoa,
Honolulu, HI 96822}

\date{\today}

\begin{abstract}
Interaction-free measurement refers to quantum phenomena such as the
presence of a photon absorber being inferred without it absorbing any
photons.  We review the physics and its realization in interferometric
experiments.  We then consider a simple classical system: two coupled
oscillators, one of which is damped; the energy dissipation can be
made arbitrarily small by making the damping large enough.  The
phenomenon can be considered analogous to interaction-free
measurement.
\end{abstract}

\maketitle

\section{Interaction-free measurement}

Many quantum paradoxes have been explored theoretically and
experimentally in recent years.  Most involve entangled states: the
famous EPR paradox is an old example, and quantum computing is a more
recent development\cite{2007qucosc.book}.  Entanglement is considered
an intrinsically quantum phenonemn classical analog.

There are also counter-intuitive quantum phenonema not involving
entanglement.  Imagine a photon, initially in a ``vertical'' polarized
state.  It is sent through a Faraday rotator, and then through a
vertically polarizing sheet.  The former rotates the polarization
plane towards ``horizontal'' by small angle, say $\pi/(2N)$, where $N$
is large.  The latter effectively measures the polarization state in
the vertical/horizontal basis; if the result comes out horizontal, the
photon is absorbed, otherwise it goes through in a vertical
eigenstate.  Since $N$ is large, the chances of being absorbed are
small.  Most likely, the photon will get slightly rotated and then
reset to vertical.  If the photon is re-routed back to the apparatus
$N$ times, it will still most likely emerge vertically polarized.  The
rotator and the polarizer appear to cancel each other out.  The
general phenomenon is known as the quantum Zeno
effect,\cite{PhysRevA.41.2295} but it can also be described in
everyday language, in terms of waking and sleeping
states.\cite{orzel2009teach}

What happens if we now remove the polarizer?  Clearly, the $N$ Faraday
rotations will take the photon to a horizontal state.  Thus, the net
effect of the polarizer is (most likely) to rotate a photon from
horizontal to vertical.  Recall that the polarizer is really just an
absorber of horizontal photons.  Yet it can, within a clever setup, be
made to rotate photons instead of absorbing them.  This is the essence
of interaction-free measurement.  Since the initial suggestion
(essentially the $N=1$ case)\cite{1993FoPh...23..987E} and
generalization to the now-standard form, together with the first
experiments\cite{PhysRevLett.74.4763} interaction-free measurement has
been the subject of many
demonstrations\cite{voorthuysen:1504,2006JPhB...39.3177N} and
reviews,\cite{deweerd:272} and applications aiming towards medical
imaging are starting to appear\cite{PhysRevA.80.040902}.  (An older
usage of similar
terminalogy\cite{springerlink:10.1007/BF01327019,dicke:925} refers to
a situation where a non-detection enables an inference.)

In this paper, we relate the above simple picture on the one hand to
the standard description of interaction-free measurement in terms of
an interferometric setup, and on the other hand to a classical system
of coupled oscillators.

\section{Interferometric picture}

Most discussion in the literature is in terms of a Mach-Zehnder
interferometer.  With the help of Figure~{\ref{fig:mz}} we can relate
this to qubit notation.
\begin{equation}
\ket0 \equiv \left( \begin{array}{c} 1 \\ 0 \end{array} \right)  \qquad
\ket1 \equiv \left( \begin{array}{c} 0 \\ 1 \end{array} \right)  \qquad
\X \equiv \left( \begin{array}{cc} 0 & 1 \\ 1 & 0 \end{array} \right)
\end{equation}
A mirror acts as a swap operator $i\X$ and a beam-splitter acts as
\begin{equation}
e^{i\alpha\X} \equiv \cos\alpha + i\sin\alpha\,\X
\end{equation}
Then $\sin^2\frac12\alpha$ is the reflectivity.

The whole interferometer applies the operation
\begin{equation}
e^{i\alpha\X} i\X e^{i\alpha\X} = -\sin2\alpha + i\cos2\alpha X
\end{equation}
which is a rotation in qubit space.

If we write
\begin{equation}
\alpha = \frac\pi2 - \epsilon
\end{equation}
then we have
\begin{equation}
-\sin2\epsilon + i\cos2\epsilon X
\end{equation}


The operation can be concatenated, say $N$ times.  Now let us select
\begin{equation}
\alpha = \frac\pi2 \left(1-\frac1N\right)
\end{equation}
If now we start with $\ket0$ and go through the interferometer $N$
times, we are certain to get $\ket0$.

Now put an absorber in one of the arms, as in Figure~\ref{fig:absorb}.
Then, after going through the interferometer $N$ times, the
probability of getting $\ket1$ will be
\begin{equation}
\sin^{4N}\alpha = \cos^{4N} \left(\frac\pi{2N}\right)
\end{equation}
which approaches unity for large $N$.

\section{Coupled oscillators}

Consider two coupled oscillators, one of them damped.
\begin{equation}
\begin{array}{l}
\ddot x + x + \kappa y = 0 \\
\ddot y + y + \kappa x + \gamma y = 0
\end{array}
\end{equation}
If $\gamma=0$ has solutions of the form
\begin{equation}
\begin{array}{l}
x = \cos(\sqrt{1+\kappa}\,t) + \cos(\sqrt{1-\kappa}\,t)  \\
y = \cos(\sqrt{1+\kappa}\,t) - \cos(\sqrt{1-\kappa}\,t)
\end{array}
\end{equation}
and the amplitude sloshes from one to the other.

With damping, easiest to solve using Laplace transform. For the
initial conditions $x(0)=1$ and $\dot x(0) = y(0) = \dot y(0) = 0$ we
have
\begin{equation}
\begin{array}{l}
s^2 X - s + X + \kappa Y = 0 \\
s^2 Y + Y + \kappa X + \gamma s Y = 0
\end{array}
\end{equation}
For strong damping, we have to $O(1/\gamma)$
\begin{equation}
\begin{array}{l}
\displaystyle
X = \frac s{s^2+1} + \frac{\kappa^2}\gamma \frac1{(s^2+1)^2} \\
\displaystyle
Y = -\frac\kappa\gamma \frac1{s^2+1}
\end{array}
\end{equation}
Solution to $O(1/\gamma)$
\begin{equation}
\begin{array}{l}
\displaystyle
x = \cos t + \frac{\kappa^2}{2\gamma} (\sin t - t\cos t) \\
\displaystyle
y = -\frac\kappa\gamma \sin t
\end{array}
\end{equation}
Amplitude damping rate $\propto\kappa^2/\gamma$.

For mechanical oscillators, it may be counter-intuitive that more
damping means less energy dissipation.  In the electrical case (an LC
circuit coupled to an RLC circuit) it is obvious that increasing
resistance reduces dissipation.

\bibliography{heap.bib}

\newpage

\def\zero{{\bf 0}}
\def\vmargin{\hrule height 1cm width 0pt}

\def\pgfbeamsplitter(#1,#2){\pgfputat{\pgfxy(#1,#2)}
                            {\pgfbox[center,center]
                             {$e^{i\alpha\X}$}
                            \pgfxyline(-.8,0)(0,.8)
                            \pgfxyline(0,.8)(.8,0)
                            \pgfxyline(.8,0)(0,-.8)
                            \pgfxyline(0,-.8)(-.8,0)}}

\def\pgfmirror(#1,#2){\pgfputat{\pgfxy(#1,#2)}
                      {\pgfbox[center,center]{$\X$}
                      \pgfxyline(-.4,-.4)(.4,-.4)
                      \pgfxyline(.4,-.4)(.4,.4)
                      \pgfxyline(.4,.4)(-.4,.4)
                      \pgfxyline(-.4,.4)(-.4,-.4)}}

\def\pgfabsorber(#1,#2){\pgfputat{\pgfxy(#1,#2)}
                      {\pgfbox[center,center]{$\zero$}
                      \pgfxyline(-.4,-.4)(.4,-.4)
                      \pgfxyline(.4,-.4)(.4,.4)
                      \pgfxyline(.4,.4)(-.4,.4)
                      \pgfxyline(-.4,.4)(-.4,-.4)}}


\begin{figure}
\begin{center}
\vmargin
\begin{pgfpicture}
\pgfbeamsplitter(0,0)
\pgfmirror(2,2)
\pgfmirror(2,-2)
\pgfbeamsplitter(4,0)
\pgfsetendarrow{\pgfarrowsingle}
\pgfxyline(-1.5,-1.5)(-.5,-.5)
\pgfxyline(.5,.5)(1.5,1.5)
\pgfxyline(.5,-.5)(1.5,-1.5)
\pgfxyline(2.5,1.5)(3.5,.5)
\pgfxyline(2.5,-1.5)(3.5,-.5)
\pgfxyline(4.5,.5)(5.5,1.5)
\pgfxyline(4.5,-.5)(5.5,-1.5)
\pgfputat{\pgfxy(-1.7,-1.7)}{\pgfbox[right,center]{$|0\rangle$}}
\pgfputat{\pgfxy(5.7,1.7)}{\pgfbox[left,center]{$i\sin2\alpha\;|0\rangle$}}
\pgfputat{\pgfxy(5.7,-1.7)}{\pgfbox[left,center]{$\cos2\alpha\;|1\rangle$}}
\end{pgfpicture}
\vmargin\vmargin\vmargin
\begin{pgfpicture}
\pgfbeamsplitter(0,0)
\pgfmirror(2,2)
\pgfmirror(2,-2)
\pgfbeamsplitter(4,0)
\pgfsetendarrow{\pgfarrowsingle}
\pgfxyline(-1.5,1.5)(-.5,.5)
\pgfxyline(.5,.5)(1.5,1.5)
\pgfxyline(.5,-.5)(1.5,-1.5)
\pgfxyline(2.5,1.5)(3.5,.5)
\pgfxyline(2.5,-1.5)(3.5,-.5)
\pgfxyline(4.5,.5)(5.5,1.5)
\pgfxyline(4.5,-.5)(5.5,-1.5)
\pgfputat{\pgfxy(-1.7,1.7)}{\pgfbox[right,center]{$\ket1$}}
\pgfputat{\pgfxy(5.7,1.7)}{\pgfbox[left,center]{$\cos2\alpha\;\ket0$}}
\pgfputat{\pgfxy(5.7,-1.7)}{\pgfbox[left,center]{$i\sin2\alpha\;\ket1$}}
\end{pgfpicture}
\end{center}
\caption{\label{fig:mz} Schematic representation of a Mach-Zehnder
  interferometer, with the results of two possible inputs.  Arrows
  corresponds to light paths.  We express a photon moving towards
  upper right as $\ket0$ and a photon moving towards lower right as
  $\ket1$.  A mirror acts as a swap operator $\X$.  A beam splitter is
  expressed by the operator $e^{i\alpha\X}$.  If $\alpha=0$ the
  beam-splitter is fully transmissive, while $\alpha=\frac\pi2$ means
  fully reflective.}
\end{figure}

\begin{figure}
\begin{center}
\vmargin
\begin{pgfpicture}
\pgfbeamsplitter(0,0)
\pgfabsorber(2,2)
\pgfmirror(2,-2)
\pgfbeamsplitter(4,0)
\pgfsetendarrow{\pgfarrowsingle}
\pgfxyline(-1.5,-1.5)(-.5,-.5)
\pgfxyline(.5,.5)(1.5,1.5)
\pgfxyline(.5,-.5)(1.5,-1.5)
\pgfxyline(2.5,-1.5)(3.5,-.5)
\pgfxyline(4.5,.5)(5.5,1.5)
\pgfxyline(4.5,-.5)(5.5,-1.5)
\pgfputat{\pgfxy(-1.7,-1.7)}{\pgfbox[right,center]{$\ket0$}}
\pgfputat{\pgfxy(5.7,1.7)}
         {\pgfbox[left,center]{$\frac i2\sin2\alpha\;\ket0$}}
\pgfputat{\pgfxy(5.7,-1.7)}{\pgfbox[left,center]{$-\sin^2\alpha\;\ket1$}}
\end{pgfpicture}
\vmargin
\end{center}
\caption{\label{fig:absorb} Schematic represenation of a Mach-Zehnder
  interferometer with an absorber in one arm.}
\end{figure}


\def\pgfbeamsplitter(#1,#2)#3{\pgfputat{\pgfxy(#1,#2)}
                              {\pgfbox[center,center]#3
                               \pgfxyline(-1.1,0)(0,1.1)
                               \pgfxyline(0,1.1)(1.1,0)
                               \pgfxyline(1.1,0)(0,-1.1)
                               \pgfxyline(0,-1.1)(-1.1,0)}}

\def\semireflecting{$\hbox{beam}\atop\hbox{splitter}$}
\def\arbireflecting{$e^{i\alpha\X}$}
\def\nearlytransparent{$e^{i\left(\frac\pi2-\epsilon\right)\X}$}

\def\pgfsimpmirror(#1,#2){\pgfputat{\pgfxy(#1,#2)}
                      {\pgfbox[center,center]{M}
                      \pgfxyline(-.4,-.4)(.4,-.4)
                      \pgfxyline(.4,-.4)(.4,.4)
                      \pgfxyline(.4,.4)(-.4,.4)
                      \pgfxyline(-.4,.4)(-.4,-.4)}}

\def\pgfmirror(#1,#2){\pgfputat{\pgfxy(#1,#2)}
                      {\pgfbox[center,center]{$i\X$}
                      \pgfxyline(-.4,-.4)(.4,-.4)
                      \pgfxyline(.4,-.4)(.4,.4)
                      \pgfxyline(.4,.4)(-.4,.4)
                      \pgfxyline(-.4,.4)(-.4,-.4)}}

\def\pgfabsorber(#1,#2){\pgfputat{\pgfxy(#1,#2)}
                      {\pgfbox[center,center]{$\zero$}
                      \pgfxyline(-.4,-.4)(.4,-.4)
                      \pgfxyline(.4,-.4)(.4,.4)
                      \pgfxyline(.4,.4)(-.4,.4)
                      \pgfxyline(-.4,.4)(-.4,-.4)}}

\begin{figure}
\begin{center}
\vmargin

\begin{pgfpicture}
\pgfbeamsplitter(0,0)\semireflecting
\pgfsimpmirror(2,2) % 1-3,6
% \pgfabsorber(2,2) % 4,5
\pgfsimpmirror(2,-2)
\pgfbeamsplitter(4,0)\semireflecting
\pgfsetendarrow{\pgfarrowsingle}
\pgfxyline(-1.5,-1.5)(-.7,-.7) % 1,3-
% \pgfxyline(-1.5, 1.5)(-.7, .7) % 2
\pgfxyline(.7,.7)(1.5,1.5)
\pgfxyline(.7,-.7)(1.5,-1.5)
\pgfxyline(2.5,1.5)(3.3,.7) % 1-3,6
\pgfxyline(2.5,-1.5)(3.3,-.7)
\pgfxyline(4.7,.7)(5.5,1.5) % 1,3-
% \pgfxyline(4.7,-.7)(5.5,-1.5) % 2,4-5
\end{pgfpicture}

\vmargin\vmargin\vmargin

\begin{pgfpicture}
\pgfbeamsplitter(0,0)\semireflecting
\pgfsimpmirror(2,2) % 1-3,6
% \pgfabsorber(2,2) % 4,5
\pgfsimpmirror(2,-2)
\pgfbeamsplitter(4,0)\semireflecting
\pgfsetendarrow{\pgfarrowsingle}
% \pgfxyline(-1.5,-1.5)(-.7,-.7) % 1,3-
\pgfxyline(-1.5, 1.5)(-.7, .7) % 2
\pgfxyline(.7,.7)(1.5,1.5)
\pgfxyline(.7,-.7)(1.5,-1.5)
\pgfxyline(2.5,1.5)(3.3,.7) % 1-3,6
\pgfxyline(2.5,-1.5)(3.3,-.7)
% \pgfxyline(4.7,.7)(5.5,1.5) % 1,3-
\pgfxyline(4.7,-.7)(5.5,-1.5) % 2,4-5
\end{pgfpicture}

\vmargin\vmargin\vmargin

\begin{pgfpicture}
\pgfbeamsplitter(0,0)\semireflecting
% \pgfsimpmirror(2,2) % 1-3,6
\pgfabsorber(2,2) % 4,5
\pgfsimpmirror(2,-2)
\pgfbeamsplitter(4,0)\semireflecting
\pgfsetendarrow{\pgfarrowsingle}
\pgfxyline(-1.5,-1.5)(-.7,-.7) % 1,3-
% \pgfxyline(-1.5, 1.5)(-.7, .7) % 2
\pgfxyline(.7,.7)(1.5,1.5)
\pgfxyline(.7,-.7)(1.5,-1.5)
% \pgfxyline(2.5,1.5)(3.3,.7) % 1-3,6
\pgfxyline(2.5,-1.5)(3.3,-.7)
\pgfxyline(4.7,.7)(5.5,1.5) % 1,3-
\pgfxyline(4.7,-.7)(5.5,-1.5) % 2,4-5
\end{pgfpicture}

\end{center}
\caption{\label{fig:mach-zehnder}}
\end{figure}

\begin{figure}
\begin{center}
\vmargin

\begin{pgfpicture}
\pgfbeamsplitter(0,0)\nearlytransparent % 2-
\pgfmirror(2,2) % 1-4
% \pgfabsorber(2,2) % 5
\pgfmirror(2,-2)
\pgfbeamsplitter(4,0)\nearlytransparent % 2-
\pgfsetendarrow{\pgfarrowsingle}
\pgfxyline(-1.5,-1.5)(-.7,-.7) % 1-2,4-
% \pgfxyline(-1.5, 1.5)(-.7, .7) % 3
\pgfxyline(.7,.7)(1.5,1.5)
\pgfxyline(.7,-.7)(1.5,-1.5)
\pgfxyline(2.5,1.5)(3.3,.7) % 1-4
\pgfxyline(2.5,-1.5)(3.3,-.7)
\pgfxyline(4.7,.7)(5.5,1.5)
\pgfxyline(4.7,-.7)(5.5,-1.5)
\pgfputat{\pgfxy(-1.7,-1.7)}{\pgfbox[right,center]{$\ket0$}} % 1-2,4-
% \pgfputat{\pgfxy(-1.7, 1.7)}{\pgfbox[right,center]{$\ket1$}} % 3
% \pgfputat{\pgfxy(5.7, 1.7)}{\pgfbox[left,center]
%         {$-\cos2\alpha\;\ket0$}}
%         \pgfputat{\pgfxy(5.7,-1.7)}{\pgfbox[left,center]
%         {$-i\sin2\alpha\;\ket1$}}}
\pgfputat{\pgfxy(5.7, 1.7)}{\pgfbox[left,center]
         {$-\sin2\epsilon\;\ket0$}}
         \pgfputat{\pgfxy(5.7,-1.7)}{\pgfbox[left,center]
         {$-i\cos2\epsilon\;\ket1$}}
% <2,4>
% \pgfputat{\pgfxy(5.7, 1.7)}{\pgfbox[left,center]
%         {$-i\cos2\epsilon\;\ket0$}}
%         \pgfputat{\pgfxy(5.7,-1.7)}{\pgfbox[left,center]
%         {$-\sin2\epsilon\;\ket1$}}}
% <3>
% \pgfputat{\pgfxy(5.7, 1.7)}{\pgfbox[left,center]
%         {$-\frac 12\sin2\epsilon\;\ket0$}}
%         \pgfputat{\pgfxy(5.7,-1.7)}{\pgfbox[left,center]
%         {$-i\cos^2\epsilon\;\ket1$}}}
% <5>
\end{pgfpicture}

\vmargin\vmargin\vmargin

\begin{pgfpicture}
\pgfbeamsplitter(0,0)\nearlytransparent % 2-
\pgfmirror(2,2) % 1-4
% \pgfabsorber(2,2) % 5
\pgfmirror(2,-2)
\pgfbeamsplitter(4,0)\nearlytransparent % 2-
\pgfsetendarrow{\pgfarrowsingle}
% \pgfxyline(-1.5,-1.5)(-.7,-.7) % 1-2,4-
\pgfxyline(-1.5, 1.5)(-.7, .7) % 3
\pgfxyline(.7,.7)(1.5,1.5)
\pgfxyline(.7,-.7)(1.5,-1.5)
\pgfxyline(2.5,1.5)(3.3,.7) % 1-4
\pgfxyline(2.5,-1.5)(3.3,-.7)
\pgfxyline(4.7,.7)(5.5,1.5)
\pgfxyline(4.7,-.7)(5.5,-1.5)
% \pgfputat{\pgfxy(-1.7,-1.7)}{\pgfbox[right,center]{$\ket0$}} % 1-2,4-
\pgfputat{\pgfxy(-1.7, 1.7)}{\pgfbox[right,center]{$\ket1$}} % 3
% \pgfputat{\pgfxy(5.7, 1.7)}{\pgfbox[left,center]
%         {$-\cos2\alpha\;\ket0$}}
%         \pgfputat{\pgfxy(5.7,-1.7)}{\pgfbox[left,center]
%         {$-i\sin2\alpha\;\ket1$}}}
% \pgfputat{\pgfxy(5.7, 1.7)}{\pgfbox[left,center]
%         {$-\sin2\epsilon\;\ket0$}}
%         \pgfputat{\pgfxy(5.7,-1.7)}{\pgfbox[left,center]
%         {$-i\cos2\epsilon\;\ket1$}}}
% <2,4>
\pgfputat{\pgfxy(5.7, 1.7)}{\pgfbox[left,center]
         {$-i\cos2\epsilon\;\ket0$}}
          \pgfputat{\pgfxy(5.7,-1.7)}{\pgfbox[left,center]
         {$-\sin2\epsilon\;\ket1$}}
% <3>
% \pgfputat{\pgfxy(5.7, 1.7)}{\pgfbox[left,center]
%         {$-\frac 12\sin2\epsilon\;\ket0$}}
%         \pgfputat{\pgfxy(5.7,-1.7)}{\pgfbox[left,center]
%         {$-i\cos^2\epsilon\;\ket1$}}}
% <5>
\end{pgfpicture}

\vmargin\vmargin\vmargin

\begin{pgfpicture}
\pgfbeamsplitter(0,0)\nearlytransparent % 2-
% \pgfmirror(2,2) % 1-4
\pgfabsorber(2,2) % 5
\pgfmirror(2,-2)
\pgfbeamsplitter(4,0)\nearlytransparent % 2-
\pgfsetendarrow{\pgfarrowsingle}
\pgfxyline(-1.5,-1.5)(-.7,-.7) % 1-2,4-
% \pgfxyline(-1.5, 1.5)(-.7, .7) % 3
\pgfxyline(.7,.7)(1.5,1.5)
\pgfxyline(.7,-.7)(1.5,-1.5)
% \pgfxyline(2.5,1.5)(3.3,.7) % 1-4
\pgfxyline(2.5,-1.5)(3.3,-.7)
\pgfxyline(4.7,.7)(5.5,1.5)
\pgfxyline(4.7,-.7)(5.5,-1.5)
\pgfputat{\pgfxy(-1.7,-1.7)}{\pgfbox[right,center]{$\ket0$}} % 1-2,4-
% \pgfputat{\pgfxy(-1.7, 1.7)}{\pgfbox[right,center]{$\ket1$}} % 3
% \pgfputat{\pgfxy(5.7, 1.7)}{\pgfbox[left,center]
%         {$-\cos2\alpha\;\ket0$}}
%         \pgfputat{\pgfxy(5.7,-1.7)}{\pgfbox[left,center]
%         {$-i\sin2\alpha\;\ket1$}}}
% \pgfputat{\pgfxy(5.7, 1.7)}{\pgfbox[left,center]
%         {$-\sin2\epsilon\;\ket0$}}
%         \pgfputat{\pgfxy(5.7,-1.7)}{\pgfbox[left,center]
%         {$-i\cos2\epsilon\;\ket1$}}}
% <2,4>
% \pgfputat{\pgfxy(5.7, 1.7)}{\pgfbox[left,center]
%         {$-i\cos2\epsilon\;\ket0$}}
%         \pgfputat{\pgfxy(5.7,-1.7)}{\pgfbox[left,center]
%         {$-\sin2\epsilon\;\ket1$}}}
% <3>
\pgfputat{\pgfxy(5.7, 1.7)}{\pgfbox[left,center]
         {$-\frac 12\sin2\epsilon\;\ket0$}}
          \pgfputat{\pgfxy(5.7,-1.7)}{\pgfbox[left,center]
         {$-i\cos^2\epsilon\;\ket1$}}
% <5>
\end{pgfpicture}


\end{center}
\caption{\label{fig:qubits}}
\end{figure}



\end{document}

